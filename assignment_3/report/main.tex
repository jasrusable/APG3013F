\documentclass{article}

\usepackage{graphicx}
\usepackage{amsmath}

\title{APG3013F Assignment 3 - Numerical methods and quality control for least squares traverse adjustments}
\date{12/05/2015}
\author{Jason David Russell - RSSJAS005}

\begin{document}

\maketitle
\pagenumbering{gobble}

\newpage
\tableofcontents

\pagenumbering{arabic}

\newpage
\section{Introduction}
The aim of this assignment is to solve a traverse least squares adjustment by Cholesky decomposition,
and to perform a sequential least squares adjustment by adding additional observations during iteration
of a solution.

\section{Background}
\subsection{Cholesky Decomposition}
There are many techniques which can be used in when performing matrix opertaions on a large scale to 
reduce the cost of computation, and to speed up computaion. For example, with calculations involving sparse 
matricies it is useful to partition matricies such that maximum multiplication of small numbers/elements/zeros
occur. Matrix decomposition is another method of matrix opertaion optomization. Matrix decompotion essentially
involves breaking down a matrix into a formulation of multiple matricies (more than one). Some other techniques are
Guass Reduction, Jacobi Method, Gauss-Jordan Method etc. The Cholesky Method is very conveniant in that 
it provides an inverse which is later used for subsequent least squares matricies. It is for this reason that
the Cholesky method is of partiular suitability when programming a least squares solution.

\subsection{Sequential least squares}
Sequential least squares is a method least squares determination whereby additional observations
are added to a calculation post computaion (after an iteration). This can be very useful when filtering or smoothing
is required (Kalman filter).

\section{Problem Statement}
\subsection{Cholesky decomposition}
Calculate a traverse solution by Cholesky decomposition when determining the x-vector of a least squares solution.

\subsection{Sequential Least Squares}
Sequentially add observations to an adjustment (Sequential least squares).

\section{Method}
The Cholesky decomposition is used as follows when solving for a least sqaures solution vector, x: 
\[ (A^t P A)x = A^t P l \]

\section{Results}
When performing a least squares solution using a program, solving for all resultant matricies, Cholesky decompostion results
in the ability to solve for these resultant matricies much more quickly and more efficiently.

\section{Conclusion}
\subsection{Cholesky Decomposition}
The Cholesky method of solving for the x vector in a least sqaures adjustment facilitates efficiency computation and 
can result in quicker computations.

\subsection{Sequential Least Sqaures}
Sequential least sqaures is very useful for smoothing and filtering applications, and is rather trivial to implement ontop
of an existing least squares program.

\end{document}
